\chapter{Pemrograman Dasar}
\section{Teori}
\begin{enumerate}
\item
Variabel merupakan tempat menyimpan data, sedangkan tipe data adalah jenis data yang tersimpan dalam variabel, Variabel bersifat mutable, artinya nilainya bisa berubah-ubah. Variabel memiliki beberapa jenis termasuk : 
	\begin{enumerate}
	\item
	Variabel global, yaitu variabel yang bisa diakses dengan semua fungsi
	\item
	Variabel local, yaitu variabel yang bisa diakses dalam fungsi dan tempat variabel berada.
	\item
	Variabel build-in, yaitu variabel yang sudah termasuk ada dalam python
	\end{enumerate}
\item
Dengan menuliskan sricpt seperti ini : \\
A=input("Cecep") \\
maka untuk menampilkan ketikan perintah dibawah ini : \\
print("halo", Cecep "Apa Kabar")
\item
Operator dasar matematika adalah : \\
	1. + (pertambahan)\\
	2. - (pengurangan)\\
	3. / (pembagian)\\
	4. x (perkalian)\\
\begin{enumerate}
\item
Untuk merubahnya menjadi integer gunakanlah kode int() \\ 
Sebagai contoh : \\
b=876 \\ integer = int(b) <sricpt konversi dari string ke integer> \\
print(integer) <untuk mencetak hasilnya> \\
\item
Untuk merubahnya menjadi string gunakanlah kode str() \\
sebagai contoh : \\
r=747 \\ string = str(r) <sricpt konversi dari integer ke string> \\
print(string) <untuk mencetak hasilnya> \\ 
\end{enumerate}
\item
Ada 2 perulangan yaitu while loop dan for loop. 
\begin{enumerate}
\item
While loop adalah perulangan yang selalu dieksekusi selama kondisi bernilai benar(true).
\item
For loop adalah perulangan yang memiliki kemampuan untuk mengulangi item dari urutan apapun seperti list atau string
\end{enumerate} 
1. Contoh sricpt pada while loop : \\
Count = 0 \\
while (count < 9): \\
print'the count is:',count \\
Count = count+1 \\
print("Good Bye!") \\ \\
2. Contoh penerapan for loop : \\
Nomer=[1,2,3,4,5] \\
For x in nomer: \\
Print(x)\\
\item
Ada 3 macam kondisi diantaranya : 
\begin{enumerate}
\item
If \\ adalah suatu kondisi yang bernilai benar atau salah, jika dalam statmennt bernilai benar akan dijalankan, tetapi sebaliknya jika statmennt bernilai salah maka tidak akan dijalankan (eror). 
\item
If - Else \\ adalah suatu kondisi bernilai benar maka statment didalam if akan dieksekusi dan jika bernilai false maka statment yang dieksekusi adalah statment didalam else
\item
If - Elif - Else \\ adalah suatu kondisi Elif, lanjutan dari percabangan  if dengan kondisi ini menyebabkan beberapa kemungkinan statment yang terjadi.
\end{enumerate}
1. Sebagai contoh script IF yang bernilai true. \\
x=1 \\
IF x >0; \\
Print("Nilai \%x adalah besar dari 0"\%x) \#Nilai 1 adalah besar dari 0 \\
Kondisi diatas bernilai true, karena nilai x(1) lebih besar dari 0. \\ \\
Sedangkan script di bawah ini merupakan contoh dari suatu kondisi bernilai false. \\
x=1 \\
IF x >2; \\
Print("Nilai \%x adalah besar dari 0"\%x) \\
Kondisi diatas bernilai false, maka outputan tidak akan muncul.\\ \\
2. Sebagai contoh script IF-Else : \\
x=1 \\
IF x >5: \\
Print("Nilai \%d adalah besar dari 5"\%x) \\
Else:
Print("Nilai \%d adalah kecil dari \%"\%x) \\ 
\# Nilai 1 adalah kecil dari 5 \\ sebaliknya, ubahlah nilai x menjadi 20.\\
x=20 \\
IF x >5:\\
Print("Nilai \%d adalah besar dari 5"\%x)\\
Else: \\
Print("Nilai \%d adalah kecil dari 5"\%x)\\\\
3. Sebagai contoh script IF ELIF ELSE : \\
x=5\\
if x <5:\\
Print("Nilai \%d adalah kecil dari 5"\%x)\\
elif x==5:\\
Print("Nilai \%d adalah sama dengan 5"\%x)\\
else: \\
Print("Nilai \%d adalah besar dari 5"\%x)\\
\item
diantara sintak-sintak eror yang ditemui antara lain :
\begin{enumerate}
\item
TypeError:unsupported operand type(s) for +:'int' and 'str' \\ penangann error ini bisa ditandai menggunakan casting operand kedua menjadi integer
\item
TypeError:can only concatenate str(not"int") to str \\ penanganan ini ditandai dengan menggunakan casting operand kedua menjadi string. 
\end{enumerate}
\item
Try Except adalah bentuk penanganan error yang terdapat dalam bahasa pemograman (python). Contoh penanganannya : \\
Setiap bilangan yang dibagi 0 akan terjadi error karena ini merupakan ketentuan dari awal dan tidak bisa dieskekusi, tetapi dengan menggunakan try except dapat ditangani walaupun akan terjadi error dibawah ini : \\
x=0\\
Try:\\
x=9/0\\
Except exception,e;\\
print e\\
print x=1 \\
Maka akan muncul peringatan error integer devision or modulo by zero 1
\end{enumerate}

\section{Ketrampilan Pemrograman}
Buat program di python dengan ketentuan:
\begin{enumerate}
\item
Berikut ini merupakan soal praktikum no 1 dengan soal sebagai berikut : \\
Buatlah luaran huruf yang dirangkai dari tanda bintang, pagar atau plus dari NPM kita.
Tanda bintang untuk NPM mod 3=0, tanda pagar untuk NPM mod 3 =1, tanda plus untuk NPM mod3=2.
Contoh sricpt : 
\begin{verbatim}
# -*- coding: utf-8 -*-
"""
Created on Tue Oct 22 21:29:47 2019

@author: lenovo
"""

print("***   ***   *******   ******  ******  ******  ******  ");
print("***   ***   **   **  **   **  **  **  **  **      **  ");
print("***   ***    ****   ********  **  **  ******  ******  ");
print("***   ***   **   **       **  **  **      **  **      ");
print("***   ***   *******       **  ******  ******  ******  ");
\end{verbatim}
\item
Berikut ini hasil script dari soal no 2 praktikum
\begin{verbatim}
# -*- coding: utf-8 -*-
"""
Created on Tue Oct 22 21:04:07 2019

@author: lenovo
"""

npm=int(input("masukan npm anda : "))
TwoLastDigit=abs(npm)%100 # modulus menetukan ambil 2 digit terakhir
for i in range(TwoLastDigit):
    print("Halo, ", npm, " Apa kabar ?")
\end{verbatim}
Maka hasil Outputnya adalah menampilkan 'Halo, 1184021 Apa kabar ?' sebanyak 21x tampilan. 
\begin{verbatim}
masukan npm anda : 1184092
Halo,  1184092  Apa kabar ?
Halo,  1184092  Apa kabar ?
Halo,  1184092  Apa kabar ?
Halo,  1184092  Apa kabar ?
Halo,  1184092  Apa kabar ?
Halo,  1184092  Apa kabar ?
Halo,  1184092  Apa kabar ?
Halo,  1184092  Apa kabar ?
Halo,  1184092  Apa kabar ?
Halo,  1184092  Apa kabar ?
Halo,  1184092  Apa kabar ?
Halo,  1184092  Apa kabar ?
Halo,  1184092  Apa kabar ?
Halo,  1184092  Apa kabar ?
Halo,  1184092  Apa kabar ?
Halo,  1184092  Apa kabar ?
Halo,  1184092  Apa kabar ?
Halo,  1184092  Apa kabar ?
Halo,  1184092  Apa kabar ?
Halo,  1184092  Apa kabar ?
Halo,  1184092  Apa kabar ?
Halo,  1184092  Apa kabar ?
Halo,  1184092  Apa kabar ?
Halo,  1184092  Apa kabar ?
Halo,  1184092  Apa kabar ?
Halo,  1184092  Apa kabar ?
Halo,  1184092  Apa kabar ?
Halo,  1184092  Apa kabar ?
Halo,  1184092  Apa kabar ?
Halo,  1184092  Apa kabar ?
Halo,  1184092  Apa kabar ?
Halo,  1184092  Apa kabar ?
Halo,  1184092  Apa kabar ?
Halo,  1184092  Apa kabar ?
Halo,  1184092  Apa kabar ?
Halo,  1184092  Apa kabar ?
Halo,  1184092  Apa kabar ?
Halo,  1184092  Apa kabar ?
Halo,  1184092  Apa kabar ?
Halo,  1184092  Apa kabar ?
Halo,  1184092  Apa kabar ?
Halo,  1184092  Apa kabar ?
Halo,  1184092  Apa kabar ?
Halo,  1184092  Apa kabar ?
Halo,  1184092  Apa kabar ?
Halo,  1184092  Apa kabar ?
Halo,  1184092  Apa kabar ?
Halo,  1184092  Apa kabar ?
Halo,  1184092  Apa kabar ?
Halo,  1184092  Apa kabar ?
Halo,  1184092  Apa kabar ?
Halo,  1184092  Apa kabar ?
Halo,  1184092  Apa kabar ?
Halo,  1184092  Apa kabar ?
Halo,  1184092  Apa kabar ?
Halo,  1184092  Apa kabar ?
Halo,  1184092  Apa kabar ?
Halo,  1184092  Apa kabar ?
Halo,  1184092  Apa kabar ?
Halo,  1184092  Apa kabar ?
Halo,  1184092  Apa kabar ?
Halo,  1184092  Apa kabar ?
Halo,  1184092  Apa kabar ?
Halo,  1184092  Apa kabar ?
Halo,  1184092  Apa kabar ?
Halo,  1184092  Apa kabar ?
Halo,  1184092  Apa kabar ?
Halo,  1184092  Apa kabar ?
Halo,  1184092  Apa kabar ?
Halo,  1184092  Apa kabar ?
Halo,  1184092  Apa kabar ?
Halo,  1184092  Apa kabar ?
Halo,  1184092  Apa kabar ?
Halo,  1184092  Apa kabar ?
Halo,  1184092  Apa kabar ?
Halo,  1184092  Apa kabar ?
Halo,  1184092  Apa kabar ?
Halo,  1184092  Apa kabar ?
Halo,  1184092  Apa kabar ?
Halo,  1184092  Apa kabar ?
Halo,  1184092  Apa kabar ?
Halo,  1184092  Apa kabar ?
Halo,  1184092  Apa kabar ?
Halo,  1184092  Apa kabar ?
Halo,  1184092  Apa kabar ?
Halo,  1184092  Apa kabar ?
Halo,  1184092  Apa kabar ?
Halo,  1184092  Apa kabar ?
Halo,  1184092  Apa kabar ?
Halo,  1184092  Apa kabar ?
Halo,  1184092  Apa kabar ?
Halo,  1184092  Apa kabar ?
\end{verbatim}
\item
Buatlah program hello word dengan input nama yang disimpan dalam sebuah variabel string bernama \textbf{NPM} dan beri luaran output berupa tiga karakter belakang dari NPM sebanyak penjumlahan tiga dijit tersebut. berikut ini merupakan script dari soal ini.
\begin{verbatim}
# -*- coding: utf-8 -*-
"""
Created on Tue Oct 22 22:19:12 2019
@author: lenovo
"""
npm=int(input("Masukan NPM : "))
key=str(npm%1000)
print("Hallo, "+str(npm)[4]+str(npm)[5]+str(npm)[6]+" Apa kabar?")
for i in range(int(str(npm)[4])+int(str(npm)[5])+int(str(npm)[6])-1):
         print("Hallo, "+str(npm)[4]+str(npm)[5]+str(npm)[6]+" Apa kabar?")
\end{verbatim}
Maka hasilnya adalah 3x muncul "Hallo, 092 Apa Kabar?" 
\begin{verbatim}
Masukan NPM : 1184092
Hallo, 092 Apa kabar?
Hallo, 092 Apa kabar?
Hallo, 092 Apa kabar?
Hallo, 092 Apa kabar?
Hallo, 092 Apa kabar?
Hallo, 092 Apa kabar?
Hallo, 092 Apa kabar?
Hallo, 092 Apa kabar?
Hallo, 092 Apa kabar?
Hallo, 092 Apa kabar?
Hallo, 092 Apa kabar?
\end{verbatim}
\item
Buatlah program hello word dengan input nama yang disimpan dalam sebuah variabel string bernama \textbf{NPM} dan beri luaran output berupa digit ketiga dari belakang dari variabel NPM. Berikut sricpt nya :
\begin{verbatim}
# -*- coding: utf-8 -*-
"""
Created on Tue Oct 22 22:43:26 2019
@author: lenovo
"""
npm=int(input("Masukan NPM : "))
key=npm%1000
str_key=str(key)
print("hello, "+str_key[0]+" Apa kabar ?")
\end{verbatim}
Maka Hasil nya adalah sebagai berikut :
\begin{verbatim}
Masukan NPM : 1184092
hello, 2 Apa kabar ?
\end{verbatim}
\item
Pada soal no 5 ini merupakan penggunanan perulangan dan kodisi. contoh penerapan dalam sricpt nya adalah sebagai berikut :
\begin{verbatim}
# -*- coding: utf-8 -*-
"""
Created on Tue Oct 22 22:47:07 2019
@author: lenovo
"""
i=0
npm=input("Masukan NPM : ")
while i<1:
    if len(npm) < 7:
        print("NPM Kurang dari 7 digit")
        npm=input("Masukan NPM : ")
    elif len(npm) > 7:
        print("NPM lebih dari 7 digit")
        npm=input("Masukan NPM : ")
    else:
        i=1
a=npm[0]
b=npm[1]
c=npm[2]
d=npm[3]
e=npm[4]
f=npm[5]
g=npm[6]
for x in a,b,c,d,e,f,g:
    print(x,end = ""),
\end{verbatim}
Maka akan keluar outputan : \\
Masukan NPM : 1184092 \\
1184092
\item
Setelah dilakukan perulangan maka seluruh variabel dijumlahkan . maka sricpt nya sebagai berikut :
\begin{verbatim}
# -*- coding: utf-8 -*-
"""
Created on Tue Oct 22 20:58:14 2019
@author: lenovo
"""
i=0
npm=input("Masukan NPM : ")
while i<1:
    if len(npm) < 7:
        print("NPM Kurang dari 7 digit")
        npm=input("Masukan NPM : ")
    elif len(npm) > 7:
        print("NPM lebih dari 7 digit")
        npm=input("Masukan NPM : ")
    else:
        i=1
a=npm[0]
b=npm[1]
c=npm[2]
d=npm[3]
e=npm[4]
f=npm[5]
g=npm[6]
y=0
for x in a,b,c,d,e,f,g:
    y+=int(x)
    print(y)
\end{verbatim}
Hasilnya adalah seperti ini :
\begin{verbatim}
Masukan NPM : 1184092
1
2
10
14
14
16
17
\end{verbatim}
\item 
Sama hal-nya pada no 6, yang membedakannya pada no 7 ini adalah veriabel tersebut dikalikan dengan aritmatikan perkalian, contohnya sricpt sebagai berikut : 
\begin{verbatim}
# -*- coding: utf-8 -*-
"""
Created on Tue Oct 22 23:02:17 2019
@author: lenovo
"""
i=0
npm=input("Masukan NPM : ")
while i<1:
    if len(npm) < 7:
        print("NPM Kurang dari 7 digit")
        npm=input("Masukan NPM : ")
    elif len(npm) > 7:
        print("NPM lebih dari 7 digit")
        npm=input("Masukan NPM : ")
    else:
        i=1
a=npm[0]
b=npm[1]
c=npm[2]
d=npm[3]
e=npm[4]
f=npm[5]
g=npm[6]
conv=1
for x in a,b,c,d,e,f,g:
    conv*=int(x)
    print(conv)
\end{verbatim}
Maka Hasilnya adalah
\begin{verbatim}
Masukan NPM : 1184092
1
1
8
32
0
0
0
\end{verbatim}
\item
Dilakukannya Proses yang outputannya vertikal, script nya berupa seperti ini :
\begin{verbatim}
# -*- coding: utf-8 -*-
"""
Created on Tue Oct 22 21:13:37 2019
@author: lenovo
"""
i=0
npm=input("Masukan NPM : ")
while i<1:
    if len(npm) < 7:
        print("NPM Kurang dari 7 digit")
        npm=input("Masukan NPM : ")
    elif len(npm) > 7:
        print("NPM lebih dari 7 digit")
        npm=input("Masukan NPM : ")
    else:
        i=1
a=npm[0]
b=npm[1]
c=npm[2]
d=npm[3]
e=npm[4]
f=npm[5]
g=npm[6]
for x in a,b,c,d,e,f,g:
    print(x)
\end{verbatim}
Maka Hasilnya adalah : 
\begin{verbatim}
Masukan NPM : 1184092
1
1
8
4
0
9
2
\end{verbatim}
\item
Menampilkan digit genap script nya adalah : 
\begin{verbatim}
# -*- coding: utf-8 -*-
"""
Created on Tue Oct 22 23:10:26 2019
@author: lenovo
"""
i=0
npm=input("Masukan NPM : ")
while i<1:
    if len(npm) < 7:
        print("NPM Kurang dari 7 digit")
        npm=input("Masukan NPM : ")
    elif len(npm) > 7:
        print("NPM lebih dari 7 digit")
        npm=input("Masukan NPM : ")
    else:
        i=1
a=npm[0]
b=npm[1]
c=npm[2]
d=npm[3]
e=npm[4]
f=npm[5]
g=npm[6]
conv=1
for x in a,b,c,d,e,f,g:
    if int(x)%2==0:
        if int(x)==0:
            x=""
    print(x,end = "")
\end{verbatim}
Maka Hasilnya adalah :
\begin{verbatim}
Masukan NPM : 1184092
8402
\end{verbatim}
\item
Dan hasil dari proses ganjil , script berupa :
\begin{verbatim}
# -*- coding: utf-8 -*-
"""
Created on Tue Oct 22 23:16:50 2019
@author: lenovo
"""
i=0
npm=input("Masukan NPM : ")
while i<1:
    if len(npm) < 7:
        print("NPM Kurang dari 7 digit")
        npm=input("Masukan NPM : ")
    elif len(npm) > 7:
        print("NPM lebih dari 7 digit")
        npm=input("Masukan NPM : ")
    else:
        i=1
a=npm[0]
b=npm[1]
c=npm[2]
d=npm[3]
e=npm[4]
f=npm[5]
g=npm[6]
conv=1
for x in a,b,c,d,e,f,g:
    if int(x)%2==1:
        print(x,end ="")
\end{verbatim}
Hasilnya adalah 
\begin{verbatim}
Masukan NPM : 1184092
111
\end{verbatim}
\item 
Untuk menampilkan bilangan prima maka dilakukannya srcipt dibawah ini :
\begin{verbatim}
# -*- coding: utf-8 -*-
"""
Created on Tue Oct 22 23:20:27 2019
@author: lenovo
"""
i=0
npm=input("Masukan NPM : ")
while i<1:
    if len(npm) < 7:
        print("NPM Kurang dari 7 digit")
        npm=input("Masukan NPM : ")
    elif len(npm) > 7:
        print("NPM lebih dari 7 digit")
        npm=input("Masukan NPM : ")
    else:
        i=1
a=npm[0]
b=npm[1]
c=npm[2]
d=npm[3]
e=npm[4]
f=npm[5]
g=npm[6]
conv=1
for x in a,b,c,d,e,f,g:
    if int(x) > 1:
        for i in range(2,int(x)):
            if (int(x) % i) == 0:
                break
            else:
                print(int(x),end =""),
\end{verbatim}
Maka hasilnya adalah :
\begin{verbatim}
Masukan NPM : 1184092
tidak muncul outputan .
\end{verbatim}
\end{enumerate}


\section{Ketrampilan Penanganan Error}
Bagian Penanganan error dari script python.
\begin{enumerate}
\item
Peringatan error yang terjadi adalah "SyntaxError: invalid syntax"
\item
Contoh srcipt yang di gunakan pada file 2err.py adalah
\begin{verbatim}
# -*- coding: utf-8 -*-
"""
Created on Thu Oct 24 10:32:47 2019
@author: lenovo
"""
q="2"
r=6
try:
    p+r
    except:
        print("errror, karena hanya bisa menggabungkan string dengan string")
\end{verbatim}
\end{enumerate}

